\chapter{Introduction}

\section{Contexte du Projet}

Dans le cadre du module \textbf{Théorie des Langages et Compilation (ID1)} à l'ENSA Al Hoceima, ce projet a pour objectif la conception et l'implémentation d'un compilateur dédié aux \textbf{Automates Finis Déterministes (AFD)}.

Un automate fini déterministe est un modèle mathématique fondamental en informatique théorique, utilisé pour reconnaître des langages réguliers. Ce projet consiste à créer un outil capable de :

\begin{itemize}
    \item \textbf{Analyser} une description textuelle d'un AFD
    \item \textbf{Valider} la cohérence et le déterminisme de l'automate
    \item \textbf{Simuler} l'exécution de l'automate sur des mots donnés
    \item \textbf{Visualiser} la structure de l'automate sous forme graphique
\end{itemize}

\section{Objectifs}

Les objectifs principaux de ce projet sont les suivants :

\begin{enumerate}
    \item \textbf{Analyse Lexicale (TP2)} : Développer un analyseur lexical capable de reconnaître les tokens du langage de description d'AFD (mots-clés, identifiants, symboles, etc.)
    
    \item \textbf{Analyse Syntaxique (TP3)} : Implémenter un analyseur syntaxique qui valide la structure grammaticale des descriptions d'automates
    
    \item \textbf{Structures de Données} : Concevoir des structures C efficaces pour représenter un AFD en mémoire
    
    \item \textbf{Validation Sémantique} : Vérifier que l'automate décrit est mathématiquement correct (déterminisme, cohérence des états et symboles)
    
    \item \textbf{Fonctionnalités Avancées} : Ajouter un simulateur d'exécution et un générateur de visualisation graphique
\end{enumerate}

\section{Technologies Utilisées}

Ce projet utilise les outils standards de construction de compilateurs :

\begin{itemize}
    \item \textbf{Flex} : Générateur d'analyseurs lexicaux
    \item \textbf{Bison} : Générateur d'analyseurs syntaxiques
    \item \textbf{Langage C} : Implémentation des structures de données et de la logique
    \item \textbf{Graphviz} : Génération de visualisations graphiques des automates
    \item \textbf{Make} : Système de compilation automatisée
\end{itemize}

\section{Organisation du Rapport}

Ce rapport est organisé comme suit :

\begin{itemize}
    \item \textbf{Chapitre 2} : Analyse Lexicale - Description du lexer et des tokens
    \item \textbf{Chapitre 3} : Analyse Syntaxique - Grammaire et parser
    \item \textbf{Chapitre 4} : Structures de Données - Représentation en mémoire
    \item \textbf{Chapitre 5} : Améliorations et Fonctionnalités Élites
    \item \textbf{Chapitre 6} : Conclusion et Perspectives
\end{itemize}
