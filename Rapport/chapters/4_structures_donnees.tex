\chapter{Structures de Données}

\section{Introduction}

Le choix des structures de données est crucial pour un compilateur efficace. Ce chapitre présente la représentation en mémoire d'un AFD, définie dans le fichier \texttt{def.h}.

\section{Structure Transition}

Une transition représente un déplacement entre deux états pour un symbole donné.

\begin{lstlisting}[caption=Structure Transition]
typedef struct Transition {
    char *source;              // Etat source
    char symbol;               // Symbole de transition
    char *destination;         // Etat destination
    struct Transition *next;   // Liste chainee
} Transition;
\end{lstlisting}

\subsection{Justification : Liste Chaînée}

Les transitions sont organisées en \textbf{liste chaînée} pour plusieurs raisons :

\begin{itemize}
    \item \textbf{Flexibilité} : Nombre de transitions inconnu à l'avance
    \item \textbf{Simplicité} : Ajout dynamique en $O(1)$ en fin de liste
    \item \textbf{Mémoire} : Pas de sur-allocation comme avec un tableau
\end{itemize}

\section{Structure Automate}

La structure principale représente un AFD complet :

\begin{lstlisting}[caption=Structure Automate]
typedef struct Automate {
    char *nom;                 // Nom de l'automate
    char **alphabet;           // Tableau de symboles
    int nb_symboles;           // Taille de l'alphabet
    char **etats;              // Tableau d'etats
    int nb_etats;              // Nombre d'etats
    char *etat_initial;        // Etat de depart
    char **etats_finaux;       // Tableau d'etats finaux
    int nb_finaux;             // Nombre d'etats finaux
    Transition *transitions;   // Liste des transitions
} Automate;
\end{lstlisting}

\subsection{Champs de la Structure}

\begin{itemize}
    \item \texttt{nom} : Identifiant unique de l'automate
    \item \texttt{alphabet} : Tableau dynamique de chaînes (symboles autorisés)
    \item \texttt{nb\_symboles} : Compteur pour parcourir l'alphabet
    \item \texttt{etats} : Tableau dynamique des noms d'états
    \item \texttt{nb\_etats} : Nombre total d'états déclarés
    \item \texttt{etat\_initial} : Pointeur vers le nom de l'état de départ
    \item \texttt{etats\_finaux} : Tableau des états acceptants
    \item \texttt{nb\_finaux} : Nombre d'états finaux
    \item \texttt{transitions} : Tête de la liste chaînée de transitions
\end{itemize}

\section{Utilisation comme Table des Symboles}

La structure \texttt{Automate} joue le rôle de \textbf{table des symboles} :

\begin{itemize}
    \item \textbf{Déclaration} : Les états et symboles sont enregistrés au moment de leur déclaration
    \item \textbf{Vérification} : Avant d'ajouter une transition, le compilateur vérifie l'existence des états et symboles
    \item \textbf{Validation sémantique} : Détection immédiate des erreurs (symbole non déclaré, état inconnu, etc.)
\end{itemize}

\section{Allocation Dynamique}

Toutes les données sont allouées dynamiquement avec \texttt{malloc()} et \texttt{realloc()} :

\begin{lstlisting}[caption=Exemple - Ajout d'un état]
void ajouter_etat(char* nom_etat) {
    if (!automate_actuel) return;
    
    // Reallouer le tableau d'etats
    automate_actuel->etats = (char**)realloc(
        automate_actuel->etats,
        sizeof(char*) * (automate_actuel->nb_etats + 1)
    );
    
    // Copier le nom de l'etat
    automate_actuel->etats[automate_actuel->nb_etats] = 
        strdup(nom_etat);
    
    automate_actuel->nb_etats++;
}
\end{lstlisting}

Cette approche permet de gérer des automates de taille arbitraire sans limite fixe.

\section{Variable Globale}

Un pointeur global \texttt{automate\_actuel} est utilisé pour accéder à l'automate en cours de construction depuis le parser :

\begin{lstlisting}
extern Automate *automate_actuel;
\end{lstlisting}

Ce design pattern simplifie les actions sémantiques de Bison, qui peuvent directement modifier la structure globale.

\section{Avantages de cette Architecture}

\begin{enumerate}
    \item \textbf{Simplicité} : Structures claires et intuitives
    \item \textbf{Efficacité} : Allocation dynamique sans gaspillage
    \item \textbf{Maintenabilité} : Facile à étendre avec de nouveaux champs
    \item \textbf{Validation} : Table des symboles intégrée pour vérifications sémantiques
\end{enumerate}
